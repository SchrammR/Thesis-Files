\chapter*{Abstract}

\section*{AbstractGerman}
Avatare in VR Umgebungen liefern Vorteile in vielen Anwedungsgebieten wie der Spiele branche oder Kollaborativem VR in der Industrie. Diese Avatare müssen aber animiert werden, damit sich der Benutzer damit Identifizieren kann. Dazu gibt es verschiedene Methoden wie Bodytracking oder Inverse Kinematics (IK).
In dieser Arbeit werden die Auswirkungen der Animation eines Avatars in einer VR Anwendung hinsichtlich des Embodiments verglichen. Dabei durchlaufen zwei Gruppen ein Spiel, bei dem sie Objekten ausweichen müssen, während sie ihren Avatar in einem Spiegel vor sich sehen. Eine Gruppe benutzt dabei lediglich drei Punkte Tracing mit IK. Der Avatar der anderen Gruppe wird durch neun getrackte Punkte animiert. 
[Ergebnisse]
[Ausblick]


\section*{AbstractEnglish}
