\chapter*{Abstract}

\section*{Abstract Deutsch}
In dieser Arbeit soll beantwortet werden, wie stark sich die motorische Kontrolle über einen selbst-Avatar in virtueller Realität auf das Embodiment (Verkörperung) von Versuchspersonen über diesen kontrollierten selbst-Avatar beeinflusst. Dafür wurde ein Versuch durchgeführt, in dem die Versuchspersonen ein Ausweichspiel absolvierten, in dem die Bewegungen des eigenen Körpers zum Einsatz kamen. Der Avatar beider Gruppen wurde mithilfe von Inverse Kinematik animiert. Durch zusätzliche Tracker an den Gliedmaßen verfügte eine der Gruppen einen höheren Grad an Kontrolle über den Avatar. Dabei wurden Daten zum Embodiment, zu der Gefühlten Arbeitsbelastung sowie zur erreichte Punktzahl in dem Spiel gesammelt. Die Ergebnisse zeigen, dass der Grad an Kontrolle über einen selbst Avatar weniger Einfluss auf das Embodiment hat, wie angenommen wird.


\section*{Abstract English}
In dieser Arbeit soll beantwortet werden, wie stark sich die motorische Kontrolle über einen selbst-Avatar in virtueller Realität auf das Embodiment (Verkörperung) von Versuchspersonen über diesen kontrollierten selbst-Avatar beeinflusst. Dafür wurde ein Versuch durchgeführt, in dem die Versuchspersonen ein Ausweichspiel absolvierten, in dem die Bewegungen des eigenen Körpers zum Einsatz kamen. Der Avatar beider Gruppen wurde mithilfe von Inverse Kinematik animiert. Durch zusätzliche Tracker an den Gliedmaßen verfügte eine der Gruppen einen höheren Grad an Kontrolle über den Avatar. Dabei wurden Daten zum Embodiment, zu der Gefühlten Arbeitsbelastung sowie zur erreichte Punktzahl in dem Spiel gesammelt. Die Ergebnisse zeigen, dass der Grad an Kontrolle über einen selbst Avatar weniger Einfluss auf das Embodiment hat, wie angenommen wird.
