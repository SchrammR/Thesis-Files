\chapter*{Abstract}

\section*{Abstract Deutsch}
In dieser Bachelorthesis soll beantwortet werden, wie stark sich die motorische Kontrolle über einen Avatar in virtueller Realität (VR) auf das Embodiment (Verkörperung) von Versuchspersonen über diesen kontrollierten Avatar beeinflusst. Dafür wurde ein Versuch durchgeführt, in dem die Versuchspersonen ein Ausweichspiel absolvierten, in dem die körpereigenen Bewegungen zum Einsatz kamen. Der Avatar beider Gruppen wurde mithilfe von Inverser Kinematik animiert. Durch zusätzliche Tracker an den Gliedmaßen verfügte eine der Gruppen einen höheren Grad an Kontrolle über den Avatar. Dabei wurden Daten zu Embodiment, zu der Gefühlten Arbeitsbelastung sowie zur erreichten Punktzahl in dem Spiel gesammelt. Die Ergebnisse zeigen, dass der Grad an Kontrolle über einen Avatar weniger Einfluss auf das Embodiment hat, wie angenommen wird.


\section*{Abstract English}
This bachelorthesis aims to answer the question, how much the agency and motor control over a self avatar in virtual reality (VR) influences the sense of embodiment over this specific avatar. An experiment was conducted where participants completed a dodging-game, which was controlled with the own Body. The Avatars of both groups where animated with inverse kinematics. One of the groups had more control over the avatar with the aid of additional trackers on each limb. Data was collected in regards to embodiment, workload and points. Results show that higher control over an self avatar has less of an impact on embodiment than it is presumed.