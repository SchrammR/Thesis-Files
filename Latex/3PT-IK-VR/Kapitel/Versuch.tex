\chapter{Versuch}
In dem Versuch soll getestet werden, ob der Nutzen von Trackern die Performanz und das Embodiment eines Nutzers gegenüber einer Inverse-Kinematik Lösung für selbst Avatare erhöht. Es wurden zwei Gruppen getestet, die miteinander verglichen werden. Eine Gruppe durchläuft das Experiment mit einem durch IK animierten Avatar. Der Avatar der anderen Gruppe wird mithilfe von sechs zusätzlichen Trackern durch die Bewegungen der Testperson animiert.

\section{Technisches Setup / Versuchsaufbau}
Das Experiment wurde im VR-Lab der Informatik Fakultät in der Hochschule Reutlingen durchgeführt. 

\section{Plan}
Komplett keinen Sound
keine Texturen
spiegel
alles auf animation runterstrippen 
kompletter fokus darauf
essenz


\section{Hardware}
Bei dem Versuch kommt die HTC Vive als HMD zum Einsatz [Modell] [Bild]. Das Headset dient dem Avatar als Ankerpunkt für die ingame Kamera. Diese ist ein wenig unter den Augen des Avatars gelegen, da so die Animation durch IK besser funktionierte.
Die Vive benötigt zwei höher gelegene Kameras, welche in dem Raum an einem Gerüst festgemacht sind. Mithilfe eines Tablets und der Osram? Software können die Kameras aus der Entfernung ein und ausgeschaltet werden. Die Kameras sind für das Tracking im Raum zuständig und decken in meinem Fall ungefähr drei Meter mal fünf Meter ab. Dieses Gebiet wird in SteamVR kalibriert und dient im Spiel als begehbares Gebiet für den Spieler. [[Computer auf dem das Game läuft]].
Dazu kommen zwei VIVE-Controller, deren Position ebenfalls von den Kameras erfasst werden. In jeder Hand wird ein Controller gehalten, daher steuert die Position der Controller die Position der Hände in der Anwendung.  Der Aufbau beider Versuchsgruppen ist bis zu diesem Punkt genau gleich.
Bei Versuchsgruppe zwei werden zusätzlich zu dem oben genannten sechs(weil 6 verfügbar) VIVE Tracker verwendet [[Bild]]. Die Konfiguration, wo die sechs Tracker angebracht werden können, variiert stark. Theoretisch gesehen können die Tracker an jeglichem Objekt oder überall am Körper festgemacht werden. Das verwendete Avatarrig von VRIK besteht aus [[ca. 30]] verschiedenen Knochen ausgenommen der Finger und Zehenknochen. Standardmäßig vorgesehene Targets für Tracker gibt es in VRIK 10. Da die HMD und die Controller bereits drei davon abdecken, konnten die möglichen Tracking Ziele auf sieben begrentzt werden. Da alle konfigurationen, die einen Tracker an der Hüfte beinhaltetetn, Probleme verursachten, konnte die optimale konfiguration für sechs Tracker festgelegt werden. [[vllt. ne quelle wo sagt dass 6 tracker gut sind]]. Die durch Gruppe zwei getrackten Körperteile sind also der Kopf durch die HMD, die Hände durch die Controller sowie jeweils beide Ellbogen, Knie und Füße mithilfe der Tracker. 
Die Tracker werden mithilfe von 1/4 Zoll Kamersastativschrauben an beidseitigen Klettbändern befestigt, welche leicht am Körper angebracht werden können. Da neben der Position auch die Rotation der Tracker relevant ist, wurden die Tracker in den Versuchen immer mit der Seite des Lichtpunkts nach unten gedreht. [[Bild von Trackerrig]]
Trotz der Tracker kommt bei Gruppe zwei IK zum Einsatz, da Knochen des Rigs wie die Hüfte, die Wirbelsäule oder die Schultern nicht getrackt werden und sich so natürlicher bewegt.

[Bild von Labor] [Bild von jemanden mit Trackern]

\section{Aufbau Game?}
Wie genau funktioniert das Game?
[Bild vom Game]
Der Spieler/Proband befindet sich in einem quadratischem Raum ohne Decke. Die komplette Wand vor dem Spieler besteht aus einem virtuellen Spiegel. Die Fläche des Raumes ist ungefähr doppelt so groß als das begehbare Gebiet des Spielers. SteamVR zeigt automatisch ein rotes Netz dort an, wo das begehbare Gebiet aufhört, damit der Spieler nicht gegen Sachen außerhalb seiner freien Fläche stößt. Zusätzlich erstellte ich zusätzlich gelbe Indikatoren für das Gebiet auf dem Boden, da das Netz von SteamVR nicht in dem Spiegel angezeigt wird. Auf dem Spiegel befindet sich eine Punkteanzeige.
Wenn die Anwendung gestartet wird, befinden sich vor dem Spieler ein Tutorial und eine Fläche zum Starten des Spiels. Das Tutorial zeigt einen roten Quader, welcher die Objekte zum Ausweichen darstellt, in Nachfolgendem \textit{Hazards} genannt. Darauf ist eine "-1", da dem Spieler für das Berühren eines roten Quaders ein Punkt abgezogen wird. Daneben befindet sich eine grüne Kugel mit der Aufschrift "+2" was die Objekte darstellt, die der Spieler einsammeln soll um pro Stück zwei Punkte zu bekommen. Die grünen Kugel werden im Nachfolgenden \textit{Collectibles} genannt.
Das Startfeld muss berührt werden damit der Durchlauf des Spiels beginnt. Die komplette Anwendung verzichtet auf Tasteneingaben des Benutzers, alle benötigten Eingaben passieren durch Berührung des Avatars mit den Objekten.
Sobald das Spiel gestartet wurde, bewegen sich von vorne aus dem Spiegel die Hazards und Collectibles in einem bestimmten Intervall und bewegen sich durch das begehbare Gebiet bis sie wieder aus der Rückwand verschwinden. Die Position der einzelnen Objekte wird vor dem Spiel zufällig innerhalb einer Range festgelegt. Zusätzlich können alle Objekte mit einer Wahrscheinlichkeit von ?prozenz statt auf dem Boden in der Luft schwebend erscheinen. Dies regt den Spieler an, sich in gewissen Situationen zu ducken um einem Hazard auszuweichen oder seine Hände zu bewegen um ein Collectible, welche über einem Hazard schwebt, einzusammeln. Während der gesamten Zeit wird dem Spieler seine Punktzahl angezeigt [[warum?]] Nachdem 40? Hazards und 20? Collectibles erschienen sind, ist das Spiel vorbei. Neben der gesamten Punktzahl werden dann die jeweils getroffenen Hazards und Collectibles angezeigt. 


\section{Implementation}
Die Anwendung wurde mithilfe von Unity 2018.3.11f erstellt. Unity ist eine Engine um Computerspiele zu entwickeln. Unity bietet über den \textit{Assetstore} die Möglichkeit, Programme von Drittanbietern leicht in die eigene Anwendung zu integrieren. Die wichtigsten Assets für die Anwendung waren \textit{FinalIK} von rootmotion\cite{rootmotion} sowie SteamVR von Valve?. FinalIK bietet vorgefertigte IK-Lösungen für eine Reihe an Anwendungsarten. Das im Experiment benutzte IK-Rig stammt von dem FinalIK Anwendungsbeispiel VRIK. [[bild von VRIK]]. Abbildung X zeigt die Standardkonfiguation der Knochen von VRIK. SteamVR bietet standardfunktionalitäten für die HTC VIVE wie das Kamerarig und die Position der Controller. Die Positionen der getrackten Objekte werden dann den Knochen Punkten zugesieen

 
\section{Aufgabe}
 Punkte
 Was für Hazards
 Wie bewegen sie sich

\section{Probanden}
- Anzahl
- Unbezahlt
- Geschlecht
- Durchschn. Alter
- Wer hat schonmal Immersives VR benutzt
- Wer hat Videospiel erfahrung

