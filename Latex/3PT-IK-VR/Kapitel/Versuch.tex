\chapter{Versuch}

Ich habe einen Versuch an X Personen durchgeführt in dem sie Dingen ausweichen mussten. Ihre Performanz wird dabei in Pukten gemessen.

\section{Hypothese}
Meine Hypothese ist, dass die höhere Immersion durch exaktere Bewegungen des Avatars erhöht wird. Ich denke dass diese höhere Immersion dabei hilft die Aufgabe besser zu absolvieren.
Worauf ist diese Hypothese basiert?

\section{Versuchsdesign}
Bei dem Versuch kommt die HTC Vive als HMD zum Einsatz [Modell] [Bild]. Dazu kommen zwei Controller, die Kameras an dem Rig im Raum. Dazu kommen 6 Vive Tracker. Jeweils einer an jedem Fuß, Knie und Ellbogen.
Jeder der Probanden absolviert den Task 2 mal, einmal mit aktivierten Trackern und einmal ohne. Ob sie immer am Körper sind überleg ich noch.
Warum diese Anzahl?
Die Controller und HMD tracken immer. Daraus ergibt sich zum ersten 3-Punktetracking.
Wenn die 6 zusätzlichen Tracker dazukommen, tracken wir an insgesamt 9 Punkten, also alles was der Avatar hergibt.
Die Vive ermöglicht das Bewegen innerhalb eines Definierten bereichs (den Kameras)

[Bild von Labor] [Bild von jemanden mit Trackern]

\subsection{Eingesetzte Hardware}
Vive und so weiter
 
\subsection{Das Avatarrig}
Der Avatar besteht aus 9 IK Punkten. Füße, Knie, Hände, Ellenbogen und der Kopf.
Punkte die nicht getrackt werden, sind durch Inverse Kinematics (IK) animiert. Dabei kommt das Tool FinalIK zum Einsatz (Warum?)
SteamVR und so
 
 \subsection{Aufgabe}
 Punkte
 Was für Hazards
 Wie bewegen sie sich

\section{Probanden}
- Anzahl
- Unbezahlt
- Geschlecht
- Durchschn. Alter
- Wer hat schonmal Immersives VR benutzt
- Wer hat Videospiel erfahrung

\section{Durchführung}
Wie genau funktioniert das Game?
[Bild vom Game]
Der Spieler/Proband befindet sich in einem geschlossenen Raum, der etwas größer als das begehbare Gebiet ist. Das Begehbare gebiet ist auf dem Boden mit Roter Linie gegeben.
Vor dem Spieler befindet sich ein Spiegel ungefähr der Größe der Wand.
Über einem wird im HUD die Punktezahl angezeigt. Wenn man Hazard berührt werden Punkte angezeigt.