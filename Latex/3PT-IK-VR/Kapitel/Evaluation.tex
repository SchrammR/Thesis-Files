\chapter{Evaluation}

\section{Fragebögen}
Die Versuchspersonen wurden jeweils vor und nach dem Experiment gebeten, einen Fragebogen auszufüllen. Der erste Fragebogen vor dem Experiment beinhaltet Fragen zur Person selbst, der zweite Fragebogen nach dem Experiment beinhaltet Fragen zu den Eindrücken der Person während des Experiments. Beide Fragebögen sind ausschließlich auf Deutsch. Die Fragen des zweiten Fragebogens wurden zur Hilfe der Verständlichkeit aus dem Englischen ins Deutsche übersetzt. Die Fragebögen lagen ausschließlich in Digitaler Form vor. Zur Erstellung der Fragebögen wurde das Online Tool Google Forms verwendet. Die Fragen der Fragebögen befinden sich im Anhang.
Der erste Fragebogen enthielt zu beginn eine schriftliche Beschreibung des Versuchs, was die Person erwartet und wie lange der Versuchs ungefähr dauern wird sowie ein Bild, welches die Sicht des Benutzers in der Anwendung zeigt. Nachdem die Benutzer zur Teilnahme am Versuch zugestimmt hat, musste die vor der Teilnahme zugewiesene Teilnehmer ID eingetragen werden. Die ID sollte dabei helfen, die Antworten der beiden Fragebögen einander zuzuordnen. Zunächst wurden die Personen nach Geschlecht und Alter gefragt. Die nächste Frage beschäftigt sich damit, ob die Testpersonen eine Sehhilfe benötigen oder eine Farbsehstörung aufweisen. Die letzte Frage des ersten Fragebogens bezieht sich auf die Vorkenntnisse der Person im Bezug auf Virtuelle Realität (VR). 
Im zweiten Fragebogen, der nach dem Experiment vorgelegt wurde, sollte als erstes die erreichte Punktzahl sowie die einzelnen Plus- und Minuspunkte während des Versuchs eingetragen werden. Als nächstes wurde gefragt, ob die Person Übelkeit während des Experiments verspürt hatte.
Im Anschluss wurden die Fragen eines Fragebogens für Avatar Embodiment in VR (VR-AEB) von Gonzalez-Franco und Peck \cite{Gonzalez-Franco2018} gestellt. Bei dem Fragebogen handelt es sich um einen Vorschlag für einen Standard Fragebogen für Embodiment, da noch kein solcher standardisierter Fragebogen um Embodiment für Avatare zu messen, existiert. Gonzalez-Franco und Peck analysierten dabei über 30 Experimente, die im Zeitraum von 1998 bis 2018 stattfanden und sich mit Embodiment auseinandersetzten. Die in diesen Experimenten gestellten Fragen wurden klassifiziert, wobei sechs Kategorien an Fragen enstanden.
Die Reihenfolge der gestellten Fragen wurde, wie in Gonzalez-Francos und Pecks Paper vorgeschlagen, Randomisiert. Jede der Fragen konnte von -3 (Ich stimme überhaupt nicht zu) bis zu +3 (Ich stimmte voll und ganz zu) beantwortet werden.
Von den 25 Fragen des Fragebogens wurden 16 gefragt, da einige Fragen nicht auf das Experiment zutreffend waren. Fragen, die sich mit Berührung auseinandersetzen sowie die komplette Kategorie \textit{Reaktion auf externe Stimuli} wurden daher nicht gefragt.
Nach dem Fragebogen zu Embodiment, wurde die Person Fragen aus dem NASA Task Load Index \cite{HART1988} gefragt. Dabei soll die wahrgenommene Belastung im Hinblick auf die Aufgabe gemessen werden.
Zuletzt wurde der Testperson die Möglichkeit gegeben, in einem freien Kommentarfeld zusätzliche Eindrücke und Kommentare zu dem Experiment zu geben.

Zusätzlich zu den Fragebögen wurde eine Liste geführt, welche Teilnehmer ID die zusätzlichen Tracker verwendet hat und welche nicht. Dazu wurden zusätzliche Kommentare der Teilnehmer erfasst, die sie während des Experiments von sich gaben und auch spezielle Beobachtungen während des Experiments.

\subsection{Embodiment Fragebogen}
Der Fragebogen zu VR-Avatar Embodiment (VR-AEB) besteht aus 25 Fragen, die in folgende Kategorien unterteilt sind:
1. Body Ownership
Zughörigkeit des Sichtbaren, virtuellen Körpers, unabhängig von dem Standort des Körpers. Diese Kategorie an Fragen wurde in 96 Prozent der Fragebögen verwendet. Die
Fragen aus dieser Kategorie sind die Fragen, die sich damit beschäftigen, ob man sich generell mit dem virtuellen Körper identifieren kann, unabhängig des Kontexts.


2. Agency and Motor Control
Misst den Grad an Kontrolle des Benutzers über den virtuellen Körper und dessen Gliedmaßen. Wiochtig bei body tracking
3. Tactile sensations
Misst, ob und wie stark sich haptische stimulation auf das Embodiment auswirkt.
4. Location of the body
relevant, da der körper sich ein bisschen von allein bewegt
5. Appearance
Äußere Erscheinung
6. Response to external stimuli
am wenigsten benutzt, vorallem wenn der Körper in gefahr ist

This questionnaire is meant to be administered at the end of
every study that involves an avatar to represent participants. It is important to emphasize how radically different the effects of the experience can be depending on the degrees of embodiment that participants experience. Aspects such as participant’s behavior or their physiology are heavily influenced by the embodiment score (González-Franco et al., 2014; Padrao et al., 2016; Slater and Sanchez-Vives, 2016). The embodiment itself can be modulated by the type, race and look of the avatar participants embody (Hershfield et al., 2011; Kilteni et al., 2013; Peck et al., 2013). Therefore, potentially any manipulation to the participants’ avatar might have strong changes on their performance during and after their VR experience.

reinbringen


\subsection{TLX}


\section{Auswertung}
Die statistische Auswertung der Daten wurde in IBM SPSS Statistics ausgeführt. 
Zu der Hypothese H1, die annimmt, dass das Embodiment bei der Gruppe mit Trackern höher ist, wird die Nullhypothese H0 formuliert. Die zu untersuchende Hypothese, die gegenüber H0 gestellt ist, wird Alternativhypothese H1 genannt.
H0 Embodiment: Der Grad an Kontrolle über einen selbst-Avatar hat keine Auswirkungen auf das Embodiment.
H1 Embodiment: Der Grad an Kontrolle über einen selbst-Avatar hat Auswirkungen auf das Embodiment.
H0 Workload: Der Grad an Kontrolle über einen selbst-Avatar hat keine Auswirkungen auf den gefühlten Grad an Belastung hinsichtlich der Aufgabe.
H1 Workload: Der Grad an Kontrolle über einen selbst-Avatar hat Auswirkungen auf den gefühlten Grad an Belastung hinsichtlich der Aufgabe.

Der p-Wert, der die Wahrscheinlichkeit angibt, mit der die Nullhypothese fälschlicherweise verworfen wird, ist als 0,05 gesetzt.
Liegt die errechnete Wahrscheinlichkeit unter dem p-Wert von 0,05 kann angenommen werden, dass das Ergebnis nicht zufällig entstanden ist. Liegt die errechnete Wahrscheinlichkeit unter einem p-Wert von 0,1 kann zumindest ein Trend angenommen werden.
Die Wahrscheinlichkeit wird berechnet, indem angenommen wird, dass die Nullhypothese H0 wahr ist. Wenn sich herausstellt, dass die Nullhypothese trotz der unterschiedlicher Versuchbedingungen wahr ist, kann angenommen werden dass die Versuchsbedungungen keinen Einfluss auf die zu untersuchenden Effekte hat. Das Ziel ist es also, die Nullhypothese zu widerlegen.
Die Berechnung der Nullhypothese erfolgt individuell für jede der Fragen in den Fragebögen.

Damit die Statistische Auswertung über einen zweiseitigen t-Test möglich ist, müssen die Ergebnisse der Experiments auf die Normalverteiluing überprüft werden. Nur wenn die Ergebnisse Normalverteilt sind, kann der T-Test durchgeführt werden. Der zweiseitige T-Test wird für Experimente verwendet, die zwei Gruppen mit den selben Fragen aufweisen und rechnet mit Mittelwerten. Für die Ergebnisse, die nicht normalverteilt sind, wird der Mann-Whitney U-Test durchgeführt. Der Mann-Whitney U-Test ist eine Version des t-Tests, der auf nicht normalverteilte Daten angewendet werden kann. Die Daten müssen dafür jedoch in Ränge eingestuft sein. Der dabei errechnete Z-Wert gibt an, ob die Wahrscheinlichkeit der Nullhypothese unter den festgelegten 5 Prozent ist. Dafür existieren Tabellen, in denen der Z-Wert nachgeschaut werden kann?????? Die Z-Verteilung mit Tabelle kann nur eingesetzt werden, wenn die Anzahl der Probanden größer als 20 ist, was im durchgeführten Experiment mit 21 Probanden der Fall war.



Statistische Signifikanz p wert unter 5%
Alles wurde auf Normalverteilung geprüft, nur drei davon sind normalverteilt -> notwendig für T test
t test ist für zwei gruppen mit den selben fragen


SE standard error

mann whitney für die die nicht normalverteilt waren

keines der dinger war statistisch signifikant


Hypothese H1 ist was ich untersuchen möchte, nämlich ob embdimtent mit tracker höher ist als ohne tracker
daraus ergibt sich sie nullhypothese H0: Das Embodiment ist bei beiden gruppen gleich

-> ich nehme an, dass H0 richtig ist
-> ich berechne die wahrscheinlichkeitsverteilung der variablen und schätze erwartungswert und standardabweichung
-> t test oder anderer berechnet ob trotz H0=richtig angenommen ob ein unterschied besteht - a
-> a wird mit p=5% verglichen  -- wenn darunter dann signifikant
-> beta wird als Teststärke oder Sensivität berechnet

ich benutze die Z verteilung da meine größe 20 ist


\section{Embodiment Ergebnisse}
Die Ergebnisse aus dem VR-Avatar Embodiment Fragebogen liefern keine statistisch signifikanten Ergebnisse. Keiner der Werte liegt innerhalb der festgelegten 0,10 (p<0,10) um überhaupt einen Trend anzuzeigen. Dennoch weisen manche Fragen eine Gemeinsamkeit der Antworten zwischen den beiden Gruppen vor. Dass keine der Daten eine statistische Signifikanz aufweisen, lässt sich auf die kleinen Größen der einzelnen Gruppen mit jeweils 10 und 11 Teilnehmern zurückführen. Ein weiterer Grund könnte sein, dass das Gefühl von Embodiment sich stark von Person zu Person unterscheidet. Dieser Unterschied spiegelt sich in den Daten wieder, da bei mehreren Fragen innerhalb einer Gruppe Werte von -3 bis +3 vorhanden sind.
Weitere mögliche Gründe sind die mangelnde Erfahrung der Teilnehmer mit VR, da nur drei der Probanden bereits mehr als einmal ein VR System nutzten und die dadurch geringe Anzahl an Vergleichsmöglichkeiten mit anderen Anwendungen, sowie die generell geringe Präsenz an selbst-Avataren in VR Anwendungen.

Auch wenn nach Abtahi \cite{Abtahi2019} die Agency eine wichtige Rolle für das Embodiment in VR Anwendungen spielt, rückt die Agency je nach Kontext weit in den Hintergrund. Da die Probanden während des Ausweichspiels trotz Spiegel keine Möglichkeiten hatten, sich in Ruhe den Avatar anzuschauen, hatte die Kontrolle über den Avatar keine nachweisbaren Auswirkungen auf das Embodiment.


agency vergleich: \cite{Abtahi2019}
die haben verglichen wie die größe des avatars sich auf die person auswirkt
in einem der drei tests war die person normal groß mit ienem auvatar auf der straße und hat die fragen aus dem agency bogen verwendet
war so bei 0,5


agency hat auswirkungen auf embodiment: \cite{Koilias2019}

obwohl in literatur agency als wichtig angesehen wird
unterschiede sind zu erkennen, jedoch nicht sehr signifikant

generell embodiment war recht hoch




body ownership ist größtenteils unbeeinflusst von agency \cite{Koilias2019} + 2 weitere
auch meine ergebnisse

agency hat aber thgeoretisch auswirkungen auf gefühlte selbst agenyc

es gab antworten, dass das game so intense war dass man gar nicht auf den Körper geachtet hat


diagramm zu den kategorien und dem total emodiment

\subsection{Diskussion - Embodiment Bogen}

\section{TLX Ergebnisse}
\cite{Abtahi2019}
tlx vergleich

\section{Performanz der Benutzer}


vr sickness kann negiert werden mit 0% dropout und nur einer person die einen geringen grad an sickness angegeben hat

\section{Kommentare}




















