\chapter{Evaluation}

\section{Fragebögen}
Die Versuchspersonen wurden jeweils vor und nach dem Experiment gebeten, einen Fragebogen auszufüllen. Der erste Fragebogen vor dem Experiment beinhaltet Fragen zur Person selbst, der zweite Fragebogen nach dem Experiment beinhaltet Fragen zu den Eindrücken der Person während des Experiments. Beide Fragebögen sind ausschließlich auf Deutsch. Die Fragen des zweiten Fragebogens wurden zur Hilfe der Verständlichkeit aus dem Englischen ins Deutsche übersetzt. Die Fragebögen lagen ausschließlich in Digitaler Form vor. Zur Erstellung der Fragebögen wurde das Online Tool Google Forms verwendet. Die Fragen der Fragebögen befinden sich im Anhang.
Der erste Fragebogen enthielt zu beginn eine schriftliche Beschreibung des Versuchs, was die Person erwartet und wie lange der Versuchs ungefähr dauern wird sowie ein Bild, welches die Sicht des Benutzers in der Anwendung zeigt. Nachdem die Benutzer zur Teilnahme am Versuch zugestimmt hat, musste die vor der Teilnahme zugewiesene Teilnehmer ID eingetragen werden. Die ID sollte dabei helfen, die Antworten der beiden Fragebögen einander zuzuordnen. Zunächst wurden die Personen nach Geschlecht und Alter gefragt. Die nächste Frage beschäftigt sich damit, ob die Testpersonen eine Sehhilfe benötigen oder eine Farbsehstörung aufweisen. Die letzte Frage des ersten Fragebogens bezieht sich auf die Vorkenntnisse der Person im Bezug auf Virtuelle Realität (VR). 
Im zweiten Fragebogen, der nach dem Experiment vorgelegt wurde, sollte als erstes die erreichte Punktzahl sowie die einzelnen Plus- und Minuspunkte während des Versuchs eingetragen werden. Als nächstes wurde gefragt, ob die Person Übelkeit während des Experiments verspürt hatte.
Im Anschluss wurden die Fragen eines Fragebogens für Avatar Embodiment in VR (VR-AEB) von Gonzalez-Franco und Peck \cite{Gonzalez-Franco2018} gestellt. Bei dem Fragebogen handelt es sich um einen Vorschlag für einen Standard Fragebogen für Embodiment, da noch kein solcher standardisierter Fragebogen um Embodiment für Avatare zu messen, existiert. Gonzalez-Franco und Peck analysierten dabei über 30 Experimente, die im Zeitraum von 1998 bis 2018 stattfanden und sich mit Embodiment auseinandersetzten. Die in diesen Experimenten gestellten Fragen wurden klassifiziert, wobei sechs Kategorien an Fragen enstanden.
Die Reihenfolge der gestellten Fragen wurde, wie in Gonzalez-Francos und Pecks Paper vorgeschlagen, Randomisiert. Jede der Fragen konnte von -3 (Ich stimme überhaupt nicht zu) bis zu +3 (Ich stimmte voll und ganz zu) beantwortet werden.
Von den 25 Fragen des Fragebogens wurden 16 gefragt, da einige Fragen nicht auf das Experiment zutreffend waren. Fragen, die sich mit Berührung auseinandersetzen sowie die komplette Kategorie \textit{Reaktion auf externe Stimuli} wurden daher nicht gefragt.
Nach dem Fragebogen zu Embodiment, wurde die Person Fragen aus dem NASA Task Load Index \cite{HART1988} gefragt. Dabei soll die wahrgenommene Belastung im Hinblick auf die Aufgabe gemessen werden.
Zuletzt wurde der Testperson die Möglichkeit gegeben, in einem freien Kommentarfeld zusätzliche Eindrücke und Kommentare zu dem Experiment zu geben.

Zusätzlich zu den Fragebögen wurde eine Liste geführt, welche Teilnehmer ID die zusätzlichen Tracker verwendet hat und welche nicht. Dazu wurden zusätzliche Kommentare der Teilnehmer erfasst, die sie während des Experiments von sich gaben und auch spezielle Beobachtungen während des Experiments.

\subsection{Embodiment Fragebogen}
Der Fragebogen zu VR-Avatar Embodiment (VR-AEB) besteht aus 25 Fragen, die in folgende Kategorien unterteilt sind:
1. Body Ownership
Zughörigkeit des Sichtbaren, virtuellen Körpers, unabhängig von dem Standort des Körpers. Diese Kategorie an Fragen wurde in 96 Prozent der Fragebögen verwendet. Die
Fragen aus dieser Kategorie sind die Fragen, die sich damit beschäftigen, ob man sich generell mit dem virtuellen Körper identifieren kann, unabhängig des Kontexts.


2. Agency and Motor Control
Misst den Grad an Kontrolle des Benutzers über den virtuellen Körper und dessen Gliedmaßen. Wiochtig bei body tracking
3. Tactile sensations
Misst, ob und wie stark sich haptische stimulation auf das Embodiment auswirkt.
4. Location of the body
relevant, da der körper sich ein bisschen von allein bewegt
5. Appearance
Äußere Erscheinung
6. Response to external stimuli
am wenigsten benutzt, vorallem wenn der Körper in gefahr ist

This questionnaire is meant to be administered at the end of
every study that involves an avatar to represent participants. It is important to emphasize how radically different the effects of the experience can be depending on the degrees of embodiment that participants experience. Aspects such as participant’s behavior or their physiology are heavily influenced by the embodiment score (González-Franco et al., 2014; Padrao et al., 2016; Slater and Sanchez-Vives, 2016). The embodiment itself can be modulated by the type, race and look of the avatar participants embody (Hershfield et al., 2011; Kilteni et al., 2013; Peck et al., 2013). Therefore, potentially any manipulation to the participants’ avatar might have strong changes on their performance during and after their VR experience.

reinbringen


\subsection{TLX}


\section{Auswertung}
Die statistische Auswertung der Daten wurde in IBM SPSS Statistics ausgeführt.
Statistische Signifikanz p wert unter 5%
Alles wurde auf Normalverteilung geprüft, nur drei davon sind normalverteilt -> notwendig für T test
t test ist für zwei gruppen mit den selben fragen


p wert ist der, der unter 0,05 sein sollte
p wert ist die signifikanz

SE standard error

mann whitney für die die nicht normalverteilt waren

keines der dinger war statistisch signifikant


\section{Embodiment Ergebnisse}
Keine aufschlussreichen ergebnisse
obwohl in literatur agency als wichtig angesehen wird
unterschiede sind zu erkennen, jedoch nicht sehr signifikant

generell embodiment war recht hoch


\subsection{Diskussion - Embodiiment Bogen}

\section{TLX}

\section{Performanz der Benutzer}

\section{Kommentare}




















