\chapter{Stand der Technik}

\section{Virtuelle Realität}
BlaBla\cite{Dummer2009}

Definition VR
Boas definiert in  seiner Arbeit \cite{Boas2012} Virtuelle Realität (VR) als ein Feld der Computerwissenschaften mit dem Ziel, immersive virtuelle Welten zu erschaffen und dem Benutzer die Möglichkeit geben, mit dieser Welt zu interagieren. Um diese Welten zu simulieren und dem Benutzer Feedback zu geben, damit die Erfahrungen so real wie möglich sind , werden bestimmte Geräte verwendet.
Der Klassische und am meisten verfügbare Weg, VR zu erleben ist ein Head-Mounted Display (HMD), ein Gerät welches um den Kopf geschnallt wird und komplett die Augen verdeckt. HMDs haben in der Regel stereoskopische Bildschirme um 3D Welten in einer großen Blickfeld darzustellen. In der Software werden dazu zwei Kameras, eine für jedes Auge, eingebunden. Durch Gyroskope und Beschleunigungssensoren erkennt das Gerät die Ausrichtung des Kopfes und kann entsprechend die Kameras in der Anwendung Bewegen. Bei manchen Ausführungen wird auch die Position des Headsets und/oder anderen Trackern erfasst. Dadurch kann sich der Benutzer im Raum bewegen und bewegt sich dadurch ebenfalls in der Anwendung. \cite{Boas2012}\cite{Holloway1995} Bei der HTC Vive beträgt die maximale Raumgröße beispielsweise 6m x 6m. ?
Aktuelle VR Kits wie die HTC Vive 2019 oder die Oculus Rift 2019 werden standardmäßig neben dem HMD mit Controllern ausgeliefert. Diese besitzen ebenfalls Tracker, mit denen ihre Position im Raum identifiziert werden kann. Der Nutzer kann mit den Controllern eine Reihe von Interaktionen Nutzen wie Aufheben und Werfen von Objekten bis zu komplexen Aufgaben wie das spannen und abfeuern eines Bogens.

Urspünglich wurde VR hauptsächlich in Militärischer, Wissenschaftlicher und Medizinischer Umgebung verwendet. Auch als Trainingwerkzeug und Fahrzeugsimulationen wurde VR eingesetzt. Heutzutage 
 Nicht nur in der Unterhaltungsbranche, sondern auch immer mehr in der Industrie.\cite{Ragan2010} *Anwendungsbeispiele*.
 
Kollaboratives VR ermöglich zusammenarbeit von Personen aus der gesamten Welt.

Erklären was ist multi user collaborative VR?

3 Seiten
\section{Immersion}

\cite{Boas2012}

\cite{Tham2018}

3 Seiten


\section{Self Embodiment}
Embodiment kann als Verkörperung übersetzt werden. Es beschreibt bla

Die Relevanz des Embodiments ist Analog zur Relevanz des eigenen Körpers in Alltäglichen Situationen. Unsere Körper liefern unserer Umgebung umgehend Informationen, wie unsere Aktivitäten, Aufmerksamkeit, Verfügbarkeit, Stimmung, Standort, Fähigkeiten und viele andere Faktoren. Der Körper kann indirekt durch Körpersprache Kommunizieren/beim Kommunizieren helfen oder allein Kommunizieren durch Zeichensprache.\cite{Benford2010}

Eine Konzeptualisierung des Embodiments ist, dass eine gemeinsamkeit aller Menschen ist, dass all unsere Erfahrungenen Grundlegend vom Beusstsein/der Bewusstheit eines Körpers beeinflusst ist.
lived experience = erlebnis
Conceptualizations of Embodiment One thing that all humans share is the fact that our lived experience is profoundly affected by our awareness of a body—physical or not. While theories of embodiment are relevant to many fields, it is most commonly defined as “how culture ‘gets under the skin,’ or the relationship of how sociocultural dynamics become translated into biological realities in the body,” according to anthropologist Anderson- Fye

Embodiment has also been discussed in relation to presence in virtual environments [20], [21], especially since there is evidence to suggest that a virtual body in the context of a head-mounted, display-based virtual reality is a critical contributor to the sense of being in the virtual location [22]. Kilteni et al. state that exploitation of immersive virtual reality has allowed a reframing of the question of felt embodiment to whether it is possible to experience the same sensations toward a virtual body inside an immersive virtual environment as toward the biological body and, if so, to what extent [23]. They offer a working definition that states that a sense of embodiment consists of three subcomponents: the sense of self-location, the sense of agency, and the sense of body ownership [23].


They offer a working definition that states that a sense of embodiment consists of three subcomponents: the sense of self-location, the sense of agency, and the sense of body ownership [23].
\cite{Tham2018}

Understanding and defining the SoE toward an ar-
tificial body can draw on ideas from recent proposals concerned with the embodiment ofartificial body parts (i.e., specific limbs), by extending these ideas to artificial whole bodies. According to de Vignemont (2011, p. 3), an object ‘‘E is embodied ifand only if some properties ofE are processed in the same way as the properties ofone’s body.’’ This definition is in line with that of Blanke and Metzinger (2009, p. 7) who state that em- bodiment includes the ‘‘subjective experience ofusing and ‘having’ a body.’’ Therefore, the following defini- tion is adopted:
SoE toward a body B is the sense that emerges when
B’s properties are processed as if they were the proper- ties ofone’s own biological body. (Definition: D)
\cite{Kilteni2012}

dann - alle drei:
sense of self location
sense of agency
sense of body ownership

dazu sind auch fragen in meinem bogen



3 Seiten

\section{Avatare}
ganz viel shit warum avatare geil sind was es bringt was damit gemacht wurde.
Damit die Benutzer den anderen Benutzern nicht als leere Hülle angezeigt werden, kommen Avatare zum Einsatz. Diese Helfen sich gegenseitig zu identifizieren und steigern zugleich das Embodiment des Nutzers selbst. 

Wichtig ist dabei wie die Körper dargestellt werden, sowohl beim eigenen Avatar als auch bei den von den anderen Mitbenutzern. Bei keinem Avatar kommt kein Embodiment zustande, andere können nur durch ihre Interaktion mit der Umgebung Identifiziert werden. Der Standard(Zitation) ist mittlerweile mindestens das HMD und die Controller + optional Hände zu sehen. \cite{Benford2010}
Soll ich hier alle Schritte der Avatare aufzählen?
Es gibt verschiedene Stufen, wie Avatare dargestellt werden können.
- Bei der Minimalsten Lösung wird kein Avatar dargestellt.Gar Kein Avatar, die Controller werden angezeigt, damit der Spieler sich im Raum orientieren kann.
- Nur Hände die die Controller halten. Hilft schon ein bisschen bei der Immersion. z.B. bei den Test Steam VR Anwendungen
- nur Arme. Gibts sowas? Bestimmt hilfreich bei onbody interfaces
- Kompletter Körper in Dummyform, einheitliche Textur. In dieser Arbeit verwende ich diese Variante, da ich mich auf die Auswirkungen der Avataranimationen fokussieren möchte. 
[Bild von meinem Dummy]
- Körper mit eigenen Maßen, Körpergröße passt. Z.B. wie beim MPI. 
- Komplett Texturiert, möglich auch mit echten Klamotten -> 3D Scanner

\section{Inverse Kinematics}







