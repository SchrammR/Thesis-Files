\chapter{Einleitung}

\section{Einführung / Hinführung zum Thema}
Motivation, Kontext und Gegenstand
Ziele

Der Anfangsteil enthält in jedem Fall die vielfach bearbeiteten Informationen aus der
ursprünglichen Aufgabenbeschreibung (TPD.2.5):
- eine Einführung in das Thema, den Kontext und Gegenstand der Arbeit, einen
Überblick zum Stand der Wissenschaft und zum Problem,
- eine genaue, vollständige und verständliche Beschreibung der Aufgabe, die Forschungsfragen und Ziele der Arbeit.

Motivation: Aus welchen sachlichen (nicht persönlichen!) Motiven und Gründen ist es
sinnvoll und nützlich, dieses Thema zu bearbeiten?

Vorgehensweise: Welche Bearbeitungsmethoden, bei empirischen Arbeiten Beobachtungs- oder Untersuchungsmethoden werden eingesetzt? Welche Lösungsansätze werden verfolgt?

Geschichte: Wie hat sich das Thema, das Fachgebiet entwickelt? Wie ordnet sich die
Arbeit in den historischen Kontext ein?

------
In dieser Arbeit geht es um animation von selbstavataren
Motivation
was will ich erreichen
wie will ichs machen

Die Forschung im Feld self-avatar animation im zusammenhang mit embodiment ist noch recht unerforscht

\section{Ziele}
Unterschied von IK zu Trackern
Bringt bessere Animation eines Avatars besseres Embodiment/Performance?
Reicht es aus nur 3-Punktetracing zu benutzen oder bringt der Mehraufwand von Trackern einen signifikanten Vorteil?

\section{Hypothese}
Meine Hypothese ist, dass die höhere Immersion durch exaktere Bewegungen des Avatars erhöht wird. Ich denke dass diese höhere Immersion dabei hilft die Aufgabe besser zu absolvieren.
Worauf ist diese Hypothese basiert?


\section{Vorgehensweise}
Wann mache ich was
in welchem Kapitel
