\chapter{Einleitung}

\section{Einführung / Hinführung zum Thema}


\section{Motivation}
Die Idee für diese Arbeit stammt von einem Projekt vom Fraunhofer IAO in Stuttgart. Bei dem Projekt handelt es sich um eine kollaborative VR-Umgebung. Dabei soll jeder Nutzer einen Avatar innerhalb des Virtuellen Raums steuern. 
Gründe dafür sind verbesserte Immersion, potenziell erhöhtes Embodiment, gesteigerte Kommunikation und erkennen des Standortes der anderen Nutzer. Vor allem in kollaborativen Umgebungen wie dieser sind Avatare wichtig, da ohne Avatare die Position und die Identität der anderen Nutzer schlecht oder gar nicht erkannt werden kann. Da das Programm auch potentiell für Meetings eingesetzt werden könnte, ist auch der Kommunikationsaspekt des potentiell verbesserten Embodiments relevant. Die Animation der Avatare funktioniert aktuell über IK, wobei die Controller an den Händen und die HMD am Kopf getrackt werden und als Referenz für den Rest des Modells dienen. Das ist zwar mit wenig Aufwand für den Nutzer verbunden, da diese Konfiguration an Hardware heutzutage der Standard bei Verbraucher orientierten komplett Systemen entspricht, jedoch sind die daraus schließenden Animationen noch ungenau und oft verzögert. Da sich die getrackten Objekte alle in der Region überhalb der Hüfte befinden, müssen ein Teil der Wirbelsäule, die Hüfte und die Beine komplett anhand der Punkte oberhalb davon animiert werden. Das führt oft zu Körpern, die bei Bewegungen dem Kopf hinterher schweben.
Eine Alternative wäre der Einsatz von mehreren Trackern an den Beinen und am Körper, womit die Genauigkeit der Animationen verbessert werden kann. Dafür führen Tracker zu höherem Entwicklungsaufwand, da sie zusätzlich IK eingesetzt werden und nicht völlig eigenständig sind. Zusätzlich liefern die Tracker je nach Anzahl einen Mehraufwand beim Nutzer, da durch sie mehr Hardware anfällt die gekauft, gelagert, aufgeladen und zum Benutzen am Körper befestigt werden muss.


\section{Ziele}
Aus den oben genannten Gründen möchte ich herausfinden, ob eine bessere Animation eines Avatars durch Tracker in VR einen Unterschied für den Benutzer in Sachen Embodiment, Immersion und Performance macht und wenn ja in welchem Ausmaß. 
Zusätzlich möchte ich herausfinden, wie groß der Mehraufwand ist, sechs Tracker zu verwenden, jeweils im Bereich der Anwendung als auch für den Nutzer vor und während der Nutzung des Programms.
Am Ende möchte ich eine Empfehlung geben, in welchen Anwendungsfällen die Nutzung von Trackern der Nutzung von IK vorzuziehen wäre und wann die Nutzung von IK der Nutzung von Trackern vorzuziehen wäre.


\section{Vorgehensweise}
Zuerst biete ich einen Überblick, wie die aktuelle Wissenschaft Virtuelle Realität (VR), Immersion, Embodiment und Avatare definiert. Danach gehe ich auf die Grundlagen des IK und Trackings ein. Im nächsten Kapitel beschreibe ich die Konzeption, Idee, Ziele und Umsetzung meines Versuchs sowie die Probleme und Herausforderungen, die Während der Entwicklung und Durchführung der Anwendung aufgetreten sind. In Kapitel 4 erkläre ich, welche Fragebögen ich ausgewählt habe, wie diese Ausgewertet wurden und zu welchem Ergebnis ich kam. Zum Schluss Gebe ich mein Fazit zu der Studie und zeige meine Zukunftsaussichten für das Thema.