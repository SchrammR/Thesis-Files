\chapter{Einführung}



\section{VR}
BlaBla\cite{Dummer2009}
Definition VR?

Virtual Reality (VR) wird immer häufiger eingesetzt. Nicht nur in der Unterhaltungsbranche, sondern auch immer mehr in der Industrie. *Anwendungsbeispiele*.
Kollaboratives VR ermöglich zusammenarbeit von Personen aus der gesamten Welt.

Erklären was ist multi user collaborative VR?

Damit die Benutzer den anderen Benutzern nicht als leere Hülle angezeigt werden, kommen Avatare zum Einsatz. Diese Helfen sich gegenseitig zu identifizieren und steigern zugleich das Embodiment des Nutzers selbst. Embodiment kann als Verkörperung übersetzt werden. Es beschreibt bla

Die Relevanz des Embodiments ist Analog zur Relevanz des eigenen Körpers in Alltäglichen Situationen. Unsere Körper liefern unserer Umgebung umgehend Informationen, wie unsere Aktivitäten, Aufmerksamkeit, Verfügbarkeit, Stimmung, Standort, Fähigkeiten und viele andere Faktoren. Der Körper kann indirekt durch Körpersprache Kommunizieren/beim Kommunizieren helfen oder allein Kommunizieren durch Zeichensprache.\cite{Benford2010}

Wichtig ist dabei wie die Körper dargestellt werden, sowohl beim eigenen Avatar als auch bei den von den anderen Mitbenutzern. Bei keinem Avatar kommt kein Embodiment zustande, andere können nur durch ihre Interaktion mit der Umgebung Identifiziert werden. Der Standard(Zitation) ist mittlerweile mindestens das HMD und die Controller + optional Hände zu sehen. \cite{Benford2010}
Soll ich hier alle Schritte der Avatare aufzählen?
- Kompletter Körper in Dummyform, einheitliche Textur. In dieser Arbeit verwende ich diese Variante, da ich mich auf die Auswirkungen der Avataranimationen fokussieren möchte. 
[Bild von meinem Dummy]
- Körper mit eigenen Maßen, Körpergröße passt
- Komplett Texturiert, möglich auch mit echten Klamotten -> 3D Scanner

\subsection{Subsection}

\subsubsection{SubSubSection}