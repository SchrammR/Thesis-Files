\chapter{Einführung}

\section{Einleitung}
Motivation, Kontext und Gegenstand
Ziele

Der Anfangsteil enthält in jedem Fall die vielfach bearbeiteten Informationen aus der
ursprünglichen Aufgabenbeschreibung (TPD.2.5):
- eine Einführung in das Thema, den Kontext und Gegenstand der Arbeit, einen
Überblick zum Stand der Wissenschaft und zum Problem,
- eine genaue, vollständige und verständliche Beschreibung der Aufgabe, die Forschungsfragen und Ziele der Arbeit.

Motivation: Aus welchen sachlichen (nicht persönlichen!) Motiven und Gründen ist es
sinnvoll und nützlich, dieses Thema zu bearbeiten?

Vorgehensweise: Welche Bearbeitungsmethoden, bei empirischen Arbeiten Beobachtungs- oder Untersuchungsmethoden werden eingesetzt? Welche Lösungsansätze werden verfolgt?

Geschichte: Wie hat sich das Thema, das Fachgebiet entwickelt? Wie ordnet sich die
Arbeit in den historischen Kontext ein?

------
In dieser Arbeit geht es um animation von selbstavataren
Motivation
was will ich erreichen
wie will ichs machen