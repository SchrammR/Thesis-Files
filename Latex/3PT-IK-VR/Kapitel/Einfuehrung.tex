\chapter{Einführung}

\section{Einleitung}
Motivation, Kontext und Gegenstand
Ziele

Der Anfangsteil enthält in jedem Fall die vielfach bearbeiteten Informationen aus der
ursprünglichen Aufgabenbeschreibung (TPD.2.5):
- eine Einführung in das Thema, den Kontext und Gegenstand der Arbeit, einen
Überblick zum Stand der Wissenschaft und zum Problem,
- eine genaue, vollständige und verständliche Beschreibung der Aufgabe, die Forschungsfragen und Ziele der Arbeit.

Motivation: Aus welchen sachlichen (nicht persönlichen!) Motiven und Gründen ist es
sinnvoll und nützlich, dieses Thema zu bearbeiten?

Vorgehensweise: Welche Bearbeitungsmethoden, bei empirischen Arbeiten Beobachtungs- oder Untersuchungsmethoden werden eingesetzt? Welche Lösungsansätze werden verfolgt?

Geschichte: Wie hat sich das Thema, das Fachgebiet entwickelt? Wie ordnet sich die
Arbeit in den historischen Kontext ein?

------
In dieser Arbeit geht es um animation von selbstavataren
Motivation
was will ich erreichen
wie will ichs machen

\section{Virtuelle Realität}
BlaBla\cite{Dummer2009}

Definition VR
Boas definiert in  seiner Arbeit \cite{Boas2012} Virtuelle Realität (VR) als ein Feld der Computerwissenschaften mit dem Ziel, immersive virtuelle Welten zu erschaffen und dem Benutzer die Möglichkeit geben, mit dieser Welt zu interagieren. Um diese Welten zu simulieren und dem Benutzer Feedback zu geben, damit die Erfahrungen so real wie möglich sind , werden bestimmte Geräte verwendet.
Der Klassische und am meisten verfügbare Weg, VR zu erleben ist ein Head-Mounted Display (HMD), ein Gerät welches um den Kopf geschnallt wird und komplett die Augen verdeckt. HMDs haben in der Regel stereoskopische Bildschirme um 3D Welten in einer großen Blickfeld darzustellen. In der Software werden dazu zwei Kameras, eine für jedes Auge, eingebunden. Durch Gyroskope und Beschleunigungssensoren erkennt das Gerät die Ausrichtung des Kopfes und kann entsprechend die Kameras in der Anwendung Bewegen. Bei manchen Ausführungen wird auch die Position des Headsets und/oder anderen Trackern erfasst. Dadurch kann sich der Benutzer im Raum bewegen und bewegt sich dadurch ebenfalls in der Anwendung. \cite{Boas2012}\cite{Holloway1995} Bei der HTC Vive beträgt die maximale Raumgröße beispielsweise 6m x 6m. ?
Aktuelle VR Kits wie die HTC Vive 2019 oder die Oculus Rift 2019 werden standardmäßig neben dem HMD mit Controllern ausgeliefert. Diese besitzen ebenfalls Tracker, mit denen ihre Position im Raum identifiziert werden kann. Der Nutzer kann mit den Controllern eine Reihe von Interaktionen Nutzen wie Aufheben und Werfen von Objekten bis zu komplexen Aufgaben wie das spannen und abfeuern eines Bogens.

Urspünglich wurde VR hauptsächlich in Militärischer, Wissenschaftlicher und Medizinischer Umgebung verwendet. Auch als Trainingwerkzeug und Fahrzeugsimulationen wurde VR eingesetzt. Heutzutage 
 Nicht nur in der Unterhaltungsbranche, sondern auch immer mehr in der Industrie.\cite{Ragan2010} *Anwendungsbeispiele*.
 
Kollaboratives VR ermöglich zusammenarbeit von Personen aus der gesamten Welt.

Erklären was ist multi user collaborative VR?

\section{Immersion}

\cite{Boas2012}

\section{Self Embodiment}

Damit die Benutzer den anderen Benutzern nicht als leere Hülle angezeigt werden, kommen Avatare zum Einsatz. Diese Helfen sich gegenseitig zu identifizieren und steigern zugleich das Embodiment des Nutzers selbst. Embodiment kann als Verkörperung übersetzt werden. Es beschreibt bla

Die Relevanz des Embodiments ist Analog zur Relevanz des eigenen Körpers in Alltäglichen Situationen. Unsere Körper liefern unserer Umgebung umgehend Informationen, wie unsere Aktivitäten, Aufmerksamkeit, Verfügbarkeit, Stimmung, Standort, Fähigkeiten und viele andere Faktoren. Der Körper kann indirekt durch Körpersprache Kommunizieren/beim Kommunizieren helfen oder allein Kommunizieren durch Zeichensprache.\cite{Benford2010}

Wichtig ist dabei wie die Körper dargestellt werden, sowohl beim eigenen Avatar als auch bei den von den anderen Mitbenutzern. Bei keinem Avatar kommt kein Embodiment zustande, andere können nur durch ihre Interaktion mit der Umgebung Identifiziert werden. Der Standard(Zitation) ist mittlerweile mindestens das HMD und die Controller + optional Hände zu sehen. \cite{Benford2010}
Soll ich hier alle Schritte der Avatare aufzählen?
Es gibt verschiedene Stufen, wie Avatare dargestellt werden können.
- Bei der Minimalsten Lösung wird kein Avatar dargestellt.Gar Kein Avatar, die Controller werden angezeigt, damit der Spieler sich im Raum orientieren kann.
- Nur Hände die die Controller halten. Hilft schon ein bisschen bei der Immersion. z.B. bei den Test Steam VR Anwendungen
- nur Arme. Gibts sowas? Bestimmt hilfreich bei onbody interfaces
- Kompletter Körper in Dummyform, einheitliche Textur. In dieser Arbeit verwende ich diese Variante, da ich mich auf die Auswirkungen der Avataranimationen fokussieren möchte. 
[Bild von meinem Dummy]
- Körper mit eigenen Maßen, Körpergröße passt. Z.B. wie beim MPI. 
- Komplett Texturiert, möglich auch mit echten Klamotten -> 3D Scanner