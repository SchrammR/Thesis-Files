\chapter{Einleitung}

\section{Einführung / Hinführung zum Thema}
Motivation, Kontext und Gegenstand
Ziele

Der Anfangsteil enthält in jedem Fall die vielfach bearbeiteten Informationen aus der
ursprünglichen Aufgabenbeschreibung (TPD.2.5):
- eine Einführung in das Thema, den Kontext und Gegenstand der Arbeit, einen
Überblick zum Stand der Wissenschaft und zum Problem,
- eine genaue, vollständige und verständliche Beschreibung der Aufgabe, die Forschungsfragen und Ziele der Arbeit.

Motivation: Aus welchen sachlichen (nicht persönlichen!) Motiven und Gründen ist es
sinnvoll und nützlich, dieses Thema zu bearbeiten?

Vorgehensweise: Welche Bearbeitungsmethoden, bei empirischen Arbeiten Beobachtungs- oder Untersuchungsmethoden werden eingesetzt? Welche Lösungsansätze werden verfolgt?

Geschichte: Wie hat sich das Thema, das Fachgebiet entwickelt? Wie ordnet sich die
Arbeit in den historischen Kontext ein?

------
In dieser Arbeit geht es um animation von selbstavataren
Motivation
was will ich erreichen
wie will ichs machen

Die Forschung im Feld self-avatar animation im zusammenhang mit embodiment ist noch recht unerforscht

\section{Motivation}
Die Motivation für diese Arbeit stammt von einem Projekt vom Fraunhofer IAO in Stuttgart. Bei dem Projekt handelt es sich um einee kollaborative VR-Umgebung. Dabei soll jeder Nutzer einen Avatar innerhalb des Virtuellen Raums steuern. 
Gründe dafür sind [potentiell] (verbesserte Immersion, höheres Embodiment), gesteigerte Kommunikation und erkennen der Position der anderen.
 Vorallem in kollaborativen Umgebungen sind Avatare wichtig, da ohne Avatare der Standort der anderen Nutzer schlecht oder gar nicht identifiziert werden kann. Da das Programm auch potentiell für Meetings eingesetzt werden könnte, ist auch der Kommunikationsaspekt des potentiell verbesserten Embodinments relevant. Dazu später mehr.
Die Animation der Avatare funktioniert aktuell über IK mit dreipunkte-tracing durch die Controller an den Händen und der HMD am Kopf. Das ist zwar mit wenig Aufwand für den Nutzer Verbunden, da diese Konfiguration an Hardware heutzutage der Standard in high fidelity HMDs ist[zitat], jedoch sind die daraus schließenden Animationen noch ungenau. Da sich die getrackten Objekte alle in der Region überhalb der Hüfte befinden, müssen ein Teil der Wirbelsäule, die Hüfte und die Beine Komplett anhand der Punkte oberhalb davon animiert werden. Das führt oft zu Körpern, die bei Bewegungen dem Kopf hinterher schweben.
Eine Alternative / zusatz wären Tracker, womit die genauigkeit der Avataranimationen verbessert werden kann. Dafür führen Tracker zu erhötem Entwicklungsaufwand, da sie anders als bei full body trackinganzügen zusätzlich zu anderen Technologien wie IK eingesetzt werden müssen. Zusätzlich bieten sie einen Mehraufwand beim Nutzer, da sie erst am Körper befestigt werden müssen und mit ihnen ein weiteres Gerät anfällt, dessen Akku geladen werden muss. Zusätzlich können zurzeit vergleichsweise teuer sein. 
(Eine VIVE mit fünf zusätzlichen Trackern kostet insgesamt doppelt so viel als ohne Tracker.)


\section{Ziele}
Aus den oben genannten Gründen möchte ich herausfinden, ob eine bessere Animation eines Avatars in VR einen Unterschied für den Benutzer in Sachen Embodiment, Immersion und Performance machen und wenn ja in welchem Ausmaß. 
Zusätzlich möchte ich herausfinden, wie groß der Mehraufwand ist, sechs Tracker zu verwenden, jeweils im Bereich der Anwendung als auch für den Nutzer vor und während der Nutzung des Programms.
Zum Schluss möchte ich eine Empfehlung geben, in welchen Anwendungsgebieten die Nutzung von Trackern die Nutzung von IK vorzuziehen wäre und umgekehrt.

Meine Hypothese ist, dass die Immersion durch exaktere Bewegungen des Avatars erhöht wird. Ich denke dass diese höhere Immersion dabei hilft die Aufgabe besser zu absolvieren.
Worauf ist diese Hypothese basiert?


\section{Vorgehensweise}
Wann mache ich was
in welchem Kapitel
