\chapter{Einleitung}

Jeder Mensch besitzt einen Körper. Ohne dass es bemerkt wird, macht jeder Mensch seinen Körper zu seinem eigenen. Doch durch den rasanten Technologischen Fortschritt, können virtuelle Welten, in denen alles möglich ist, geschaffen und erlebt werden. Das schlüpfen in den Körper einer anderen Person ist seit es Videospiele gibt, kein abstrakter Gedanke mehr, sondern Realität. Virtuelle Realität (VR) geht dabei einen Schritt weiter und ermöglicht es Menschen, tiefer in eine virtuelle Welt einzutauchen. Durch das verkörpern eines Avatars kann jeder sein was er will. Motion Tracking Technologien sind zwar teuer, erlauben es aber nicht nur den eigenen Avatar fernzusteuern, sondern ihn mit den eigenen Bewegungen zu steuern. Auch in der Software Sparte entwickelt sich das alles weiter. Inverse Kinematik bla.
VR Systeme werden immer immersiver und geben den Menschen eine immer stärkere Illusion, in einem anderen Körper in einer anderen Welt zu sein. Diese Phänomene, Embodiment und Presence, werden durch verschiedene Faktoren beeinflusst. Einer davon ist es, wie genau man seinen Avatarkörper mit den echten Bewegungen steuern kann. Da Tracking Systeme aber viel Platz brauchen, teuer und arbeitsintensiv sind, und programmatisch animierte Avatare durch IK sich immer realer bewegen, ist es interessant zu wissen, wie stark sich die Bewegunsfreiheit in einer VU auf das Embodiment auswirkt. 


\section{Motivation}
Die Idee für diese Arbeit stammt von einem Projekt des Fraunhofer IAO in Stuttgart. Bei dem Projekt handelt es sich um eine kollaborative Virtualle Realitäts (VR)-Umgebung. Dabei soll jeder Nutzer einen Avatar innerhalb des Virtuellen Raums steuern. 
Gründe dafür sind verbesserte Immersion, potenziell erhöhtes Embodiment, gesteigerte Kommunikation und leichteres Erkennen des Standortes der anderen Nutzer. Vor allem in kollaborativen Umgebungen sind Avatare wichtig, da ohne Avatare die Position und die Identität der anderen Nutzer schlecht oder gar nicht erkannt werden kann. Da das Programm auch potentiell für Meetings eingesetzt werden könnte, ist der Kommunikationsaspekt des potentiell verbesserten Embodiments ebenfalls relevant. Die Animation der Avatare funktioniert aktuell über Inverse Kinematik (IK), wobei die Position der Controller an den Händen als auch die Position des Head Mounted Displays (HMDs) mit Kameras erfasst werden und als Referenz für den Rest des Modells dienen. Diese Konfiguration an Hardware ist mit wenig Aufwand für den Nutzer des Systems verbunden, da eine HMD und zwei Controller heutzutage den Standard bei Verbraucher orientierten VR komplett Systemen entspricht und nur ein Gerät, das HMD, am Körper befestigt werden muss. Jedoch sind die aus IK mit drei Referenzpunkten entstehenden Animationen noch ungenau und oft verzögert. Da sich die von dem Kameras erfassten Objekte alle in der Region oberhalb der Hüfte befinden, müssen ein Teil der Wirbelsäule, die Hüfte sowie die Beine komplett anhand der drei Punkte oberhalb der Hüfte animiert werden. Diese Ungleichheit der Verteilung führt oft zu Körpern, die bei Bewegungen verzögert hinter dem Kopf schweben.
Eine Alternative wäre der Einsatz von mehreren zusätzlichen Trackern an den Beinen und am Körper, womit die Genauigkeit der Animationen verbessert werden kann. Dafür führen Tracker zu höherem Entwicklungsaufwand, da sie zusätzlich zu IK eingesetzt werden und nicht völlig eigenständig sind. Zusätzlich liefern die Tracker je nach Anzahl einen Mehraufwand beim Nutzer, da durch sie mehr Hardware anfällt die gekauft, gelagert, aufgeladen und zur Benutzung am Körper befestigt werden muss.


\section{Ziele}
In dieser Arbeit soll untersucht werden, wie stark sich der Grad an Kontrolle über einen Avatar und die somit verbesserten Animationen des Avatars auf das Gefühl des Embodiments des Benutzers  sowie den Grad an Erfolg in Bewegungsorientierten Aufgaben auswirkt. Dazu soll ein Versuch erstellt werden, der die Möglichkeiten IK und Motion-Tracking der Avataranimation hinsichtlich des Embodiments und der Performanz des Nutzers gegenüberstellt.


\section{Vorgehensweise}
Zuerst soll ein Überblick über den Stand der Forschung gegeben werden, in dem die Begriffe VR, Immersion, Embodiment, Presence, IK und Avatar definiert und dem Kontext der Arbeit eingeordnet werden. Im darauf folgenden Kapitel wird der im Zuge dieser Arbeit durchgeführte Versuch beschrieben. Es wird erklärt welche Entscheidungen getroffen wurden und wie der Versuch technisch umgesetzt und durchgeführt wurde. Zum Schluss werden die eingetzten Fragebögen und deren Auswertung sowie die Diskussion der Ergebnisse beleuchtet.