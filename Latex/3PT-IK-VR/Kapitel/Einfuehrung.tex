\chapter{Einleitung}

Jeder Mensch besitzt einen Körper und ohne dass es bemerkt wird, macht jeder Mensch diesen Körper zu seinem eigenen. Durch Technologien wie virtuelle Realität (VR) sind Menschen nicht mehr nur auf den eigenen Körper beschränkt. Es ist möglich geworden, in virtuelle Welten einzutauchen und dort einen vollkommen anderen, virtuellen Körper anzunehmen. Durch das verkörpern eines Avatars kann jeder sein was er will. Verschiedene Technologien wie Motion Tracking und inverse Kinematik (IK) erlauben es,  einen Avatar mit den körpereigenen Bewegungen zu steuern. 
Durch den rasanten Technologischen Fortschritt erscheinen virtuelle Welten immer realistischer, obwohl die Möglichkeiten darin unendlich sind. Zusätzlich werden VR Systeme durch bessere Hardware immer immersiver und bieten die Möglichkeiten für eine immer stärkere Illusion, in einem anderen Körper in einer anderen Welt zu sein. Diese Phänomene, Embodiment und Presence, werden unter anderem durch die genannten Faktoren wie die technologische Kapazität der VR Systeme und das Vorhandensein eines Avatars beeinflusst. Es ist aber nicht geklärt, wie groß der Einfluss des Grads an Kontrolle über den Avatar über die Illusionen des Embodiments und der Presence sind. Zum einen könnte die noch eingeschränkte aber vielversprechende programmatische Lösung über IK zur Animation von Avataren ausreichen. Zum anderen sind Tracking Systeme teuer und brauchen viel Platz, liefern aber sehr gute Ergebnisse und werden immer bezahlbarer und kleiner. Daraus erschließt sich die interessante Frage, ob der Einsatz von Motion Capture dem Einsatz von IK hinsichtlich dem Gefühl des Embodiments vorzuziehen ist.


\section{Motivation}
Die Idee für diese Arbeit stammt von einem Projekt des Fraunhofer IAO in Stuttgart. Bei dem Projekt handelt es sich um eine kollaborative VR-Umgebung. Dabei soll jeder Nutzer einen Avatar innerhalb des Virtuellen Raums steuern. 
Gründe, die für den Einsatz von Avataren sprechen sind verbesserte Immersion, potenziell erhöhtes Embodiment, gesteigerte Kommunikation und leichteres Erkennen des Standortes der anderen Nutzer. Vor allem in kollaborativen Umgebungen sind Avatare wichtig, da ohne Avatare die Position und die Identität der anderen Nutzer schlecht oder gar nicht erkannt werden kann. Da das Programm auch potentiell für Meetings eingesetzt werden könnte, ist der Kommunikationsaspekt des potentiell verbesserten Embodiments ebenfalls relevant. Die Animation der Avatare funktioniert aktuell über IK, wobei die Position der Controller an den Händen als auch die Position des Head Mounted Displays (HMDs) mit Kameras erfasst werden und als Referenz für den Rest des Modells dienen. Diese Konfiguration an Hardware ist mit wenig Aufwand für den Nutzer des Systems verbunden, da eine HMD und zwei Controller heutzutage dem Standard bei Verbraucher orientierten VR komplett Systemen entspricht und dabei nur ein Gerät, das HMD, am Körper befestigt werden muss. Jedoch sind die aus IK mit nur drei Referenzpunkten entstehenden Animationen noch ungenau und oft verzögert. Da sich die von dem Kameras erfassten Objekte alle in der Region oberhalb der Hüfte befinden, müssen ein Teil der Wirbelsäule, die Hüfte sowie die Beine komplett anhand der drei Punkte oberhalb der Hüfte animiert werden. Diese Ungleichheit der Verteilung führt oft dazu, dass der virtuelle Körper bei Bewegungen verzögert hinter dem Kopf schwebt.
Eine Alternative wäre der Einsatz von mehreren zusätzlichen Trackern an den Beinen und am Körper, womit die Genauigkeit der Animationen verbessert werden kann. Der Einsatz von Trackern führt jedoch zu höherem Entwicklungsaufwand, da sie zusätzlich zu IK eingesetzt werden und nicht völlig eigenständig eingesetzt werden können. Zusätzlich verursachen die Tracker je nach Anzahl einen höheren Aufwand beim Nutzer, da durch sie mehr Hardware anfällt die gekauft, gelagert, aufgeladen und zur Benutzung am Körper befestigt werden muss.


\section{Ziele}
In dieser Arbeit soll untersucht werden, wie stark sich der Grad an Kontrolle über einen Avatar und die somit verbesserten Animationen des Avatars auf das Gefühl des Embodiments des Benutzers, sowie den Grad an Erfolg in Bewegungsorientierten Aufgaben, auswirkt. Dazu soll ein Versuch erstellt werden, der die Möglichkeiten von IK und Motion-Tracking der Avataranimation hinsichtlich des Embodiments und der Performanz des Nutzers gegenüberstellt.


\section{Vorgehensweise}
Zuerst soll ein Überblick über den Stand der Forschung gegeben werden, in dem die Begriffe VR, Immersion, Embodiment, Presence, IK und Avatar definiert und in den Kontext der Arbeit eingeordnet werden. Im darauf folgenden Kapitel wird der im Zuge dieser Arbeit durchgeführte Versuch beschrieben. Es wird erklärt welche Entscheidungen getroffen wurden und wie der Versuch technisch umgesetzt und durchgeführt wurde. Zum Schluss werden die eingetzten Fragebögen und deren Auswertung sowie die Diskussion der Ergebnisse beleuchtet.