\chapter{Diskussion und Ausblick}
In dieser Bachelorthesis sollte der Zusammenhang des Grads an Kontrolle über einen Avatar in Virtuelle Realität (VR) und dem Embodiment über den Avatar untersucht werden.
Dazu wurde ein Versuch erstellt, in dem die Teilnehmer in einer virtuellen Umgebung (VU) mit der HTC VIVE durch Bewegungen ein Ausweichspiel absolvierten. Nach dem Versuch wurden Fragen über Embodiment und Arbeitsbelsatung in Fragebögen gestellt. 
Der Versuch nutzte zwei Versuchsbedingungen. Der Unterschied zwischen den Bedingungen war die Anzahl an erfassten Punkten, die der Animation des Avatars dienten. Realisiert wurde dieser Unterschied durch das verwenden von zusätzlichen VIVE Trackern zu dem Head Mounted Display und den Controllern der VIVE in der zweiten Bedingung.



Man kann sagen, dass Agency generell eine wichtige rolle für embodiment spielt, jedoch von anderen Faktoren überdeckt wird
gerade wenn kein Spiegel vorhanden ist und der nutzer was zu tun hat wie laufen usw ist die agency weniger wichtig, da man beim fokussieren auf einen task sich keine gedanken eber den eigenen körper macht

solange also kein spiegel vorhanden ist und oder der fokus nicht auf dem eigenen körper liegt, ist agency weniger wichtig als andere Faktoren

\section{Fazit}

\section{Zukunftsaussichten}


man könnte das selbe nochmal ohne spiegel durchführen und diese ergebnisse vergleichen

auch vergleich von chilligem kontext zu intensen kontext
eben das sitzen innerhalb eines VR meetings im vergleich mit dem erfüllen einer Aufgabe
