\chapter{Diskussion und Ausblick}
In dieser Bachelorthesis sollte untersucht werden, ob und wie stark sich der Grad an Kontrolle über einen Avatar in virtueller Realität (VR) auf das Embodiment auswirkt. Dazu wurde ein Versuch erstellt, in dem die Teilnehmer mit einer HTC VIVE ein durch Bewegungen gesteuertes Ausweichspiel absolvierten. Nach dem Versuch wurden Fragen hinsichtlich des gefühlten Embodiments und der Arbeitsbelsatung in Fragebögen gestellt. 
Der Versuch wurde mit zwei Versuchsbedingungen ausgeführt. Der Unterschied zwischen den Bedingungen war die Anzahl an erfassten Punkten, die zur Animation und Steuerung des Avatars dienten. Realisiert wurde der Unterschied zwischen den Bedingungen durch VIVE Tracker, die in einer der Bedingungen zusätzlich zu dem Head Mounted Display und den Controllern der VIVE verwendet wurden.

Die Ergebnisse zeigen keinerlei Unterschiede zwischen den beiden Versuchsbedingungen hinsichtlich des Embodiments, der Arbeitsbelastung und der Erreichten Punktzahl in dem Spiel. Das Embodiment wurde mithilfe des von Peck erstellten Vorschlag für einen standartisierten Avatar Embodiment Fragebogens gemessen, die Arbeitsbelastung unter verwendung des NASA TLX. Die Punktzahl wurde während des Spiels gespeichert. Der Versuch wurde mit einem Inter-Subjekt Design durchgeführt, die Teilnehmer kannten also die jeweils andere Versuchsbedingung nicht und konnten somit ihre Antworten auf ihre eigenen Gefühle und Erlebnisse während des Experiments beziehen. Dabei waren keine der ausgewerteten Daten statistisch signifikant, was an der geringen Gruppengröße von insgesamt 21 Personen liegen könnte.

Ein Grund für die erzielten Ergebnisse könnte sein, dass der Grad an Kontrolle (Agency) über den Avatar weniger Auswirkungen auf das Embodiment hat als angenommen wird. Agency spielt dennoch eine wichtige Rolle für das Embodiment für einen Avatar, was an den hohen Embodiment Werten sichtbar wird, obwohl als Avatar eine Bewegliche Holzpuppe eingesetzt wurde. Die Animationen des Avatars der Versuchsgruppe ohne zusätzliche Tracker, die komplett anhand drei Punkten durch Inverse Kinematics (IK) entstanden sind, lieferten überraschend zufriedenstellende Animationen. Eine Möglichkeit wäre also, dass der Unterschied zwischen den Gruppen nicht groß genug war, da die Gruppe mit weniger Agency bereits einen sehr hohen Grad an Kontrolle über den Avatar hatte. Da die Gruppen nichts voneinander wussten, konnten die beiden Arten der Kontrolle und Animation nicht im Kontext von den Teilnehmern verglichen werden, was möglicherweise zu anderen Ergebnissen geführt hätte.
In Änhlichen Untersuchungen wie in PAPER wurde herausgefunden, dass das Maß an Agency solange starken Einfluss auf das Embodiment hat, solange man sich auf den Avatarkörper und dessen Bewegungen fokussiert. Sobald eine Person eine Aufgabe zu erledigen hat, die Fokus benötigt,  spielt die Agency keine oder nur eine sehr kleine Rolle hinsichtlich des Gefühls des Embodiments. Da in dem Versuch körperliche und zum Teil auch geistige Aktivität erforderlich war, wurde die Aufmerksamkeit der Teilnehmer trotz Spiegel von dem Avatar weg auf die Aufgabe gelenkt. Hätten die Probanden mehr Zeit vor und nach der Versuchsdurchführung in der virtuellen Umgebung verbracht, wäre der Unterschied zwischen den Gruppen womöglich höher.

TLX

Die durschnittliche Punktzahlen der Gruppen war mit 21,6 und 22,9 von höchstens 40 erreichbaren Punkten ebenfalls sehr ähnlich, da der höhere Grad an kontrolle in dem Spiel wenig Auswirkungen hatte. Zwar konnte die Gruppe mit Trackern ihre Beine und Füße präziser bewegen, waren dadurch aber nicht automatisch schneller. Die größe der Objekte, denen ausgewichen werden musste, wurde so gewählt, dass keine unfairen Unterschiede zwischen den Gruppen ausweichen. So versuchten Probanden aus beiden Gruppen über die Hindernisse zu springen, welche aber in jedem Fall zu hoch dafür waren.





Zusammenfassend kann gesagt werden, dass ab einem bestimmten Grad an Kontrolle über einen Avatar der Unterschied zu mehr Kontrolle hinsichtlich des Embodiments wenig Einfluss hat. Insbesondere in Umgebungen, in denen sich eine Person bewegen, denken und Aufgaben erledigen muss, hat der hohe Grad an Kontrolle keinen messbaren Einfluss auf das Embodiment. Die Nutzung von Trackern an Armen und Beinen zusätzlich zur HMD und den Controllern der HTC VIVE erweist sich als Arbeitsintensiv und fehleranfällig für einen geringen Vorteil im Gegensatz zum Gebrauch von IK. Dennoch spielt Agency eine große Rolle für das Gefühl von Embodiment, wird abhängig von dem Kontext aber von anderen Faktoren überdeckt.


\section{Zukunftsaussichten}
Die Ergebnisse werden am Fraunhofer Institut für weitere Untersuchungen mit einbezogen.

Weitere Mögliche Versuche hinsichtlich des Themas wäre das Aufteilen der Versuchsbedingungen mit dem Vorhandensein und nicht vorhandensein eines virtuellen Spiegels. Weitere Interessante Versuche wäre der Vergleich des Einflusses der Agency auf Embodiment zwischen einem ruhigen und einem intensiven Kontext mit sonst gleichen Bedingungen. Weiter könnte die Anzahl der Tracker weiter erhöht werden, bis hin zu einem kompletten MoCap Anzug und deren Vergleich zu verschiedenen IK Lösungen.