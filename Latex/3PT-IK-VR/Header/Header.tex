%Schriftgroesse, Layout, Papierformat, Art des Dokumentes
\documentclass[listof=totoc,	12pt, oneside,	bibliography=totoc, a4paper, parskip=full+, final, headinclude=false, footinclude=false]{scrreprt} 

%Einstellungen der Seitenraender, Zeilenabstand
\usepackage[left=4cm,right=3cm,top=3cm,bottom=2cm,bindingoffset=0cm]{geometry}
\usepackage{titlesec}                  % Ueberschriftenformatierungen 
\usepackage{setspace} 
\setlength{\footskip}{0.8cm}
%bestimmt abstand zwischen footnotes und text
\setlength{\skip\footins}{6mm}

\usepackage{pdfpages}
%anpassung des Abstadnes zwischen zwei Footnotes
\setlength {\footnotesep}{0mm} 
%\enlargethispage{1.5cm}
%für subusubsection Nummerierungen
\setcounter{tocdepth}{3}
\setcounter{secnumdepth}{3} 

\usepackage{trfsigns}
%Für timing diagramme
\usepackage{tikz-timing}[2009/05/15]
\def\degr{${}^\circ$}

%% Zeilenabstand
\onehalfspacing                                     % Zeilenabstand 1,5 fach
%\KOMAoptions{DIV=last}                  % Satzspiegel neu berechnen wegen 1,5 fach Zeilenabstand
%\flushbottom 

%Deutsche Rechtschreibung für Text und Lit.-verzeichnis
\usepackage[ngerman]{babel}
\usepackage[german]{babelbib} 
%zur korrekten Schreiben vn Anführungszeichen
\usepackage[babel,german=quotes]{csquotes}
%Schriftart ändern 
%\renewcommand{\familydefault}{\sfdefault}
\usepackage{mathptmx}
%pacakge für mathematische Transormationszeichen
\usepackage{trfsigns}
%BiblatexStyle alphadin = name+Erscheinungsjahr
\bibliographystyle{unsrt}

%bibs als Section
\makeatletter
\renewcommand*\bib@heading{%
	\section*{\bibname}%
	\@mkboth{\leftmark}{\rightmark}%
}
\makeatother
\usepackage{relsize}
\usepackage{lipsum}

\newcommand{\CSharp}{C\nolinebreak[4]\hspace{-.05em}\raisebox{.4ex}{\relsize{-2}{\textbf{\#}}}}

% Verändert Bildunterschriften
\addtokomafont{caption}{\small \it} 

%Kein Einrücken bei neuem Absatz
\setlength\parindent{0pt} 

%für Abkürzungsverzeichnis
\usepackage{acronym}

%Fuer die Bibliothek
\usepackage{cite}

\usepackage[center]{caption}

% Für Symbole
\usepackage{textcomp}

% Damit einzelne Seiten im Querformat angezeigt werden können
\usepackage{pdflscape}

%Text in meheren Spalten
\usepackage{multicol}
%Um Tabellen in der tabular Umgebung zu nutzen
\usepackage{tabularx}

%Mathematische Zeichen wie Menge, Element usw
\usepackage{dsfont}


%for Tabulares
\newcolumntype{C}[1]{>{\centering\arraybackslash}p{#1}}

%Umlaute ermoeglichen
\usepackage[utf8]{inputenc}
\usepackage[T1]{fontenc}
\usepackage{lmodern}
\usepackage{textcomp}

%Fuer Grafiken
\usepackage{graphicx} 

%Um mehrere Bilder nebeneinander darstellen zu koennen
\usepackage{subfigure}

%für umrandete Texte -> Sperrvermerk
\usepackage{framed}

%Um Grafiken neben dem Text einzufuegen
\usepackage{wrapfig}

%Fuer die eingabe mathematischer Formeln
\usepackage{amsmath, amsthm, amssymb}
\usepackage{mathtools}

%Bildunterschriften manipulieren
\captionsetup{format = plain}
\addto\captionsngerman
{
	\renewcommand{\figurename}{Abb.} %Fuer die Bildunterschriften wird nun Abb. x verwendet
	\renewcommand{\tablename}{Tab.} 
	%\renewcommand{\listtablename}{List.}
}


%Kopf- und Fusszeile bzw eigene Designs, hier sind Änderungen einzutragen
\usepackage[automark]{scrpage2}


\newpagestyle{WissDokuNorm}
{ %
	(0pt,0pt)
	{\hfill\headmark}
	{\headmark\hfill}
	{\rlap{}\hfill%
		\llap{\headmark\hfill}}
	(\textwidth,.4pt)
}
{
	(\textwidth,.4pt)
	{\pagemark\hfill}
	{\hfill\pagemark}
	{Masterprojekt -Final Report-\hfill\pagemark}
	(0pt,0pt)
}

\newpagestyle{WissDokuChapter}
{ %
	(0pt,0pt)
	{\hfill}
	{\hfill}
	{\rlap{}\hfill%
		\llap{\hfill}}
	(0pt,.4pt)
}{%
(\textwidth,.4pt)
{\pagemark\hfill}
{\hfill\pagemark}
{Masterprojekt -Final Report-\hfill\pagemark}
(0pt,0pt)
}

\newpagestyle{WissDokuNothing}
{ %
	(0pt,0pt)
	{\hfill}
	{\hfill}
	{\rlap{}\hfill%
		\llap{\hfill}}
	(0pt,.4pt)
}{%
(0pt,0pt)
{\pagemark\hfill}
{\hfill\pagemark}
{\hfill\pagemark}
(0pt,0pt)
}

%Zuweisungen der gewünschten styles
\assignpagestyle{\chapter}{WissDokuChapter}
\assignpagestyle{\section}{WissDokuNorm}
\assignpagestyle{empty}{WissDokuNothing}


\pagestyle{WissDokuNorm}

%----------------------------------------------

%Für korrektes anzeigen des Quellcodes

\usepackage[utf8]{inputenc}
\usepackage{color}
\definecolor{darkblue}{rgb}{0,0,.6}
\definecolor{darkred}{rgb}{.6,0,0}
\definecolor{darkgreen}{rgb}{0,.6,0}
\definecolor{lightblue}{rgb}{0.97,0.99,1}

\usepackage{listings}
\renewcommand{\lstlistingname}{Codeauszug}
\lstdefinestyle{sharpc}{language=[Sharp]C, frame=lr,morekeywords={get,set,DateTime,List<DataBundle>,List,Exception,try,catch,ExceptionLog,DataBundle,PointF,Graphics,Control}}
\lstset
{
	style=sharpc,
	basicstyle=\ttfamily,
	commentstyle=\color{darkgreen},
	keywordstyle=\bfseries\color{darkblue},
	stringstyle=\color{darkred},
	showspaces=false,
	showstringspaces=false,
	showtabs=false,
	columns=fixed,
	frame=single,
	frameround=ffff,
	numbers=left,
	numberstyle=\tiny,
	numbersep=5pt,
	breaklines=true,
	backgroundcolor=\color{lightblue},
	captionpos=b
}
%------------------------------------------------

%für hochgestellte verweise
%\usepackage{overcite} 
%\renewcommand\citeform[1]{[#1]}




%Abkuerzungs und Symbolverzeichnis
\usepackage[german]{nomencl} %Fuer die Verzeichnisse, intoc damit es im Inhaltsverzeichnis erscheint
\usepackage{ifthen} %If then Anweisungen nutzen
\makenomenclature %Erstelle das Verzeichnis
%Definiere 2 Verzeichnis
\renewcommand{\nomname}{formeln}

%\nomenclature{Dose}{Ein wunderbares Behältnis}   Eintrag ins Abkaerzungsverzeichnis
%\nomenclature[s]{$\pi$}{Hat irgendwas mit Dosen zu tun} Eintrag ins Symbolverzeichnis
%\printnomenclature um alles auszugeben :)



\author{Thomas Gulde}
\title{ Integration einer 2D-on-the-fly-Teileerkennung in ein bestehendes Bin Picking System}



%lädt das Paket zum erzwingen der Grafikposition
\usepackage{here} 

% ermöglich es, mit \ScaleIfNeeded Grafik nur dann skalieren zu lassen, wenn sie größer als eine seite sind

\makeatletter
\def\ScaleIfNeeded{%
	\ifdim\Gin@nat@width>\linewidth
	\linewidth
	\else
	\Gin@nat@width
	\fi
}
\makeatother

% Keine "Schusterjungen"
\clubpenalty = 10000

% Keine "Hurenkinder"
\widowpenalty = 10000
\displaywidowpenalty = 10000

\raggedbottom

\setlength{\parskip}{3ex plus 3ex minus 3ex}

\renewcommand*\chapterheadstartvskip{\vspace{-\topskip}}
\renewcommand*{\chapterheadendvskip}{\vspace*{0pt}}      % Abstand zwischen Chapter u. Fliesstext
%\titlespacing{\section}{0pc}{5pt}{0pt}[0pt]
%\titlespacing{\subsection}{0pc}{5pt}{0pt}[0pt]
%\titlespacing{\subsubsection}{0pc}{5pt}{0pt}[0pt] 

%Um Links im Dokument zu aktivieren, einfach auskommentieren wenn es stoert
\usepackage{hyperref}
\hypersetup{
	colorlinks=false,
	pdfborder={0 0 0},
}

%fürs Hintergrundbild
\usepackage{eso-pic}
\newcommand\BackgroundPic{%
	\put(0,0){%
		\parbox[b][\paperheight]{\paperwidth}{%
			\vfill
			\centering
			\includegraphics[width=\paperwidth,height=\paperheight,%
			keepaspectratio]{Bilder/Deckblatt/Background.JPG}%
			\vfill
		}}}
		
		
		
		
		
		
		
		
		
		
		
		
		
		
		
		
		